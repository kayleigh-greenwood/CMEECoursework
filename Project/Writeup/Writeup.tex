\documentclass[11pt, a4paper, titlepage]{article}

% Load packages: spacing
\usepackage[onehalfspacing]{setspace} 
\usepackage{fullpage} %2.5cm margins, uses space from headers and footers

% Load packages: font
\usepackage{helvet}
\renewcommand{\familydefault}{\sfdefault}
\usepackage[svgnames,x11names,table]{xcolor} % More colour options for hyperlinks

% Load packages: graphics
\usepackage{graphicx} %enhanced graphics
\graphicspath{./Images/}
\usepackage{float} % places float in precise location in text

% Load packages: formatting
\usepackage{titlesec} % title formatting
\usepackage{lineno} % add line numbers
\usepackage[round]{natbib} % enhanced referencing
\usepackage[margin=2cm]{geometry} % sets margins
\usepackage{hyperref} % For in text links
\usepackage{caption} % Customising captions
\usepackage[labelfont=bf,textfont=bf]{caption} % Makes figure title bold

% Further document setup
\linenumbers % add line numbers
\onehalfspacing % setup line spacing
\hypersetup{
	colorlinks,
	linkcolor=DarkBlue, %colour of contents links
	citecolor=DarkBlue,
	urlcolor=DarkBlue
}



\begin{document}
    \begin{titlepage}
    \begin{center}
            {\large IMPERIAL COLLEGE LONDON}
    \end{center}
    
    \vspace*{\fill}
    
    \begin{center}
        {\Huge 
    	 Geographical Variations in the Sensitivity of Terrestrial Biodiversity to Anthropogenic Pressures}
        \\[2in]
        \begin{center}
        Author: Kayleigh Greenwood, MSc CMEE (kg21@ic.ac.uk)
        \end{center}
        \bigskip

       Internal Supervisor: Dr James Rosindell, Imperial College London (j.rosindell@imperial.ac.uk)
       \bigskip

       \begin{center}
        External Supervisor: Dr Joss Wright, University of Oxford (joss.wright@oii.ox.ac.uk)
       \end{center}
        \bigskip



        25/08/2022
        \\[2in]
        \begin{center}
        {\bfseries A thesis submitted in partial fulfilment of the requirements for the degree of Master of Science at Imperial College London}
        \end{center}
    
        \begin{center}
        {\bfseries Submitted for the MSc in Computational Methods in Ecology and Evolution }
        \end{center}
        

    
	\end{center}
    \vspace{\fill}
    
    \end{titlepage}
	\section*{Declaration}

	Data was obtained from existing online databases, and therefore I was not responsible for data collection. All data analysis and modelling is my own, with all code written by me. No mathematical models were developed for this project.

	\newpage
	
	\section*{Abstract}
	 Understanding the pressures that act on biodiversity is important in combating the acceleration of extinction rates. Global models of pressures can help to understand where biodiversity is most at threat, but often neglect to include explicit geographical variation in the sensitivity of biodiversity to those pressures. With a better understanding of geographical variation in sensitivity, the accuracy of global models could be improved. Any continental differences in sensitivity could be due to variation in biome sensitivity, as tropical biomes are more sensitive than temporal and boreal biomes. I hypothesized that the biodiversity of continents with a higher proportion of tropical biomes (e.g South America) is more sensitive than biodiversity of continents with lower proportions of tropical biomes (e.g. Europe, North America). I tested sensitivity to four main pressures on biodiversity: climate change, land use change, pollution and invasive species. Climate change is the only pressure for which sensitivity varied amongst continents. The patterns in climate change sensitivity supported the hypothesis (with Europe being less sensitive than other continents and South America being more sensitive), however no support for the hypothesis was found in the other three biodiversity pressures. These results suggest that continent-wide differences in climate change sensitivity could exist and more in-depth research (perhaps aggregating sensitivity of species by continent), could help determine whether geographical variation in sensitivity should be included in future global models to improve predictions of impact. A better ability to predict the impact of pressures can aid biodiversity-conscious decision making and by extension, reduce biodiversity loss. % but how
	 % Ask james about "if it werent understudied" section and "these results suggest" section
	 % Ask james which bits I should cut
	 % there are many reasons that continental differences in sensitivity could exist, including x y z
	 % mention more than just biomes for reasons for differences. 
	 % could finish with applications, relate to previous work (find in discussion)
	
\newpage
\tableofcontents
\newpage
	
    \section*{Introduction}
    \addcontentsline{toc}{section}{Introduction}
    	
    %Para 1: biodiversity is important. pressures act on biodiversity
	Ecosystems with intact biodiversity provide services such as clean air and pollination, which make the earth habitable for humans \citep{leemans2003millennium}. Biodiversity loss diminishes ecosystem productivity \citep{duffy2017biodiversity} and threatens all life on earth, including human well-being \citep{diaz2006biodiversity} leading to unstable environments which are less resistant to change. Both natural and anthropogenic pressures act on biodiversity \citep{nobel2020anthropogenic}, but it it is the latter which is accelerating extinction rates dramatically \citep{ceballos2015accelerated}. Understanding the impacts of anthropogenic pressures on biodiversity is necessary to inform more effective policies, strategies and tools \citep{diaz2006biodiversity, hansen2001global}. This understanding is aided by an understanding of the  distribution of biodiversity, how its pressures are distributed, and understanding how sensitive biodiversity is to these pressures. The worlds' biodiversity is not equally distributed, it varies geographically \citep{gaston2000global, ricklefs2004comprehensive, mcrae2017diversity}  as do the pressures acting on it \citep{millennium2005ecosystems, sala2000global, bowler2020mapping}, and these distributions often correlate \citep{ament2019compatibility, Velde2022}. Despite any such correlations between biodiversity and its pressures, there will always be variation in biodiversity response to such pressures due to variations in sensitivity of species \citep{bowler2020mapping}. 
	
   	\phantomsection
   	\subsection*{Interspecific Variation in Sensitivity}
   	\addcontentsline{toc}{subsection}{Interspecific Variation in Sensitivity} 
   	 %Para: Introducing species varying sensitivities and why they aren't enough
   	 Interspecific variation exists in response to biodiversity pressures \citep{foden2013identifying}. Given that both species and higher taxa \citep{sunday2015species} have shown variations in sensitivity, and each region of the world comprises different combinations of species groups \citep{goethem2021biodiversity}, there is reason to believe that sensitivity to biodiversity pressure varies depending on location. Despite this, geographical variation in sensitivity is rarely accounted for \citep{newbold2020tropical, sala2000global}. \citet{newbold2020tropical} stated that by including the sensitivity of individual species, distribution models implicitly capture geographical variation of sensitivity, but rarely explicitly include geographical variation in sensitivity. To the best of my knowledge there is no research to support this exclusion of geographical variation in sensitivity. An understanding of the sensitivity of biodiversity of an entire region could be a more accurate metric for studies of a broad scale. \newline
   	 \phantomsection
   	 \subsection*{Sensitivity Variation at Broader Levels}
   	 \addcontentsline{toc}{subsection}{Sensitivity Variation at Broader Levels} 
   	 Biodiversity sensitivity varies at broader levels than species and higher taxon, such as biomes and regions (discussed below), further supporting the concept that sensitivity variation could exist at levels as broad as continent. Biodiversity sensitivity differs between biomes for the following biodiversity pressures; pollution (nitrogen exceedance) \citep{alkemade2009globio3}, land use change and climate change \citep{newbold2020tropical}. The most sensitive biomes were tropical biomes \citep{barlow2016anthropogenic} with the least sensitive being temperate and boreal biomes \citep{newbold2020tropical, cazalis2021mismatch, barlow2016anthropogenic}. Despite geographical variations in sensitivity existing between biomes, little is known about continental differences. Biomes are unequally distributed between the continents, with South America having a higher percentage cover of tropical forest than any other continent, and Europe having the lowest closely followed by North America \citep{wade2003distribution}.  Because of the aforementioned sensitivity differences between biomes, it could be that Europe and North America will have lower sensitivity to biodiversity pressures than other continents, with South America being the most sensitive. However, of the minimal research that has studied continental differences, findings showed tropical forest biodiversity sensitivity to be higher in Asia than in other regions (Americas and Africa) \citep{gibson2011primary}. Despite the apparent contradiction with the above prediction (South America being the most sensitive), it must be noted that it was only Asia's tropical biome that was shown to be the highest among continents, not the biodiversity of the continent on the whole, and also not all continents were considered.
   	 
   	 One of the papers which studied interspecific sensitivity to environmental pressures of European species \citep{louette2010bioscore}, suggested country-level differences in sensitivity variation. Using species distribution data, the proportion of sensitive species in each country was used to map the effects of a change in each biodiversity pressure on the biodiversity of Europe. The tool (`Bioscore') produces a map of impact across Europe in response to changes in each of the biodiversity pressures, which suggests that even if the magnitude of a biodiversity pressure is constant across Europe, biodiversity response can still vary according to country, due to varying sensitivity of the species within such country. Though the map produced by the tool appears to show variations in sensitivity between countries, the tool is outdated and the data and code are inaccessible, with no values for country level sensitivities having been published. The BioScore tool's predictions support the concept that country-wide differences in sensitivity could exist, however an up to date, wider-breadth study is necessary to observe worldwide variances in countries sensitivities to biodiversity pressures. 
   	 
	\phantomsection
   	\subsection*{Knowledge Gap}
   	\addcontentsline{toc}{subsection}{Reason for gap in knowledge} 
 	A possible explanation for the inadequate research on sensitivity distribution is that sensitivity studies typically use a method of determining the sensitivity of individual species (obtained from published literature about individual species' responses to change), and then mapping these sensitivity values to look for trends \citep{louette2010bioscore}. Because of the intensity of this method, and the under-representation of certain regions \citep{collen2008tropical}, a global sensitivity analysis looking at regional biodiversity sensitivity would be very labour-intense. Additionally, most distribution studies focus on the most important pressures \citep{ferrier2016summary}. This study aims to use an alternate method, as described further below, using biodiversity and pressure trends at the regional level to look for geographic patterns in sensitivity to the five main biodiversity pressures (climate change, land use change, pollution, overexploitation and invasive species). Having regional information on biodiversity sensitivity will allow the impact of pressures on biodiversity to be better estimated in situations where it is not clear which species are being impacted, and only the area of impact is known. Rather than just including how biodiversity in general is likely to respond, it would be more useful to know how local biodiversity in that specific region is likely to respond. 
   	 

   	\phantomsection
   	\subsection*{Research aims}
   	\addcontentsline{toc}{subsection}{Research aims}
   	
   	%Para: Conclusion
   	 Given that anthropogenic impact on the environment is worldwide \citep{plumptre2021might}, the question should be raised of whether the geographic location of biodiversity pressures affects their impact on global biodiversity. The understanding of variations in biodiversity sensitivity, along with many other aspects of biodiversity, has knowledge gaps which desperately need filling \citep{pereira2012global}. If such geographic differences exist, they should be taken into account when attributing biodiversity-related merit to investments. To widen the scope of impact outside of the Benchmark for Nature project, taking into account geographic variations in sensitivity to biodiversity pressures could make estimates about biodiversity impact more accurate. Better understanding of biodiversity pressures will aid a better understanding of the implications of investments (and other policies) on natural ecosystems.
   	 
   	%para: Aims
 	The aim of this project is to investigate whether sensitivity of biodiversity to pressures varies geographically. I will amalgamate country-level data to look at continental differences in the main pressures on biodiversity (and say which ones). I hypothesise that continents with a higher proportion of tropical biomes (e.g. South America) will have higher sensitivity scores than continents with a lower proportion of tropical biomes (e.g Europe and North America) across all pressures.  \newpage
 	
 	% i will test 4 hypotheses, one for each of the pressures, which all hypothesize that biodiversity sensitivity all following the same 
 	% say i hypothesize that each pressure will... . to test this I will test the four pressures independently.
 	% i hypothesise that countries with a higher proportion
	% why is this hypothesis important to study? what is the motivation
    \section*{Methods}
	\addcontentsline{toc}{section}{Methods}
	 % Julia emphasises JUSTIFY everything you do in the results, preferably with biological reason

	The focus of this study is on anthropogenic biodiversity pressures only (for reasons outlined in the introduction). Anthropogenic pressures on biodiversity are typically grouped, in the current literature, into 5 main pressures; climate change, land use change, pollution, invasive species and overexploitation \citep{watson2019summary}. Though there are more, these are the most pervasive and therefore the most important to study \citep{mazor2018global}. 
	
	The most similar study to this one \citep{louette2010bioscore} used an alternative method of gathering information from both the literature and experts about the sensitivity of individual species (within various species groups) to the pressures on biodiversity. Focal species were selected, as it is impracticable to include every species in Europe (the geographical scope of their project). This gave each species a sensitivity score, and this data coupled with information on species distributions were used to make inferences about countries on the whole. 
	Though rigorous, the \cite{louette2010bioscore} method was extremely labour intensive, and aggregating responses of species has limitations. One limitation is that this method makes assumptions about species interactions (that there are none) which are not sufficiently understood \citep{hansen2001global}. Another limitation links to intra-specific variation. It is common practice, such as in the IUCN redlist, to extrapolate findings from a single population's sensitivity to a whole species consisting of multiple populations \citep{iucn2001iucn} \citep{buckley2012functional}. High intraspecific variation exists in response to biodiversity pressures like climate change \citep{mclean2018high} \citep{both2004large} \citep{mayor2016assessing}. This could mean that extrapolating sensitivity findings about one population of a species to the species on the whole could be problematic, and lead to unrepresentative species sensitivities.
	
	In order to avoid such limitations, and also because of the labour intensity of obtaining sensitivity scores for each species, I used an alternative method. In order to assess whether sensitivity to each pressure varies by country, I analysed relationships between annual averages of biodiversity and its' main pressures in each country. I needed data in the form of time series (how each of these pressures had been changing in each country over time, as well as how each country's biodiversity had been changing over time). I compared the time series of biodiversity in a country to the time series of a pressure on biodiversity in that country, in order to extract a `sensitivity score' for each country to assess any effect of geography. Due to comparisons between countries, I only included terrestrial biodiversity in this study. 	 % Get James' opinion


	It is important to look at individual biodiversity pressures, as opposed to an aggregated `pressure' on biodiversity, because the individual pressures have spatial differences \citep{steffen2015planetary}, meaning the geographical magnitude of each pressure varies. Therefore in order to understand how countries differ in their responses to biodiversity pressures, it must be taken into account the magnitude of each pressure that each country experiences. I modelled each pressure's geographic relationship with biodiversity in isolation as opposed to using a multiple linear model because each pressure dataset came from different sources and therefore had data from different years. Each pressure dataset has to be aligned with the biodiversity dataset to find common years, therefore if this was to be done for all pressures together, the dataset would be drastically reduced and therefore also the statistical power.
	
	I used R for all data wrangling, analysis, modelling and plotting in my project. I used linear models on time series data about biodiversity and its five main pressures at the country level, and interpret the gradient coefficients as the sensitivity' of that country's biodiversity to each pressure. I will further analyse these sensitivity scores to look for geographic variation in sensitivity between the continents. I used R version 3.6.3 for statistical analysis and plotting, and reported results as statistically significant is p $\leq$ 0.05. This method is designed to give broad insights into whether biodiversity trends differ among continents. \newline
	% get James' opinion
	\phantomsection
	\subsection*{Data}
	\addcontentsline{toc}{subsection}{Data}
	% Julia says describe data structure - eg how much data in each continent. maybe add this data in the boxplots
	% BD data = 18 years, 240 countries
	I chose biodiversity intactness as the variable to represent biodiversity. I chose the National History Museum's (NHM) Biodiversity Intactness Index (BII)\citep{phillips2021} as it presents biodiversity in the context of how many original species remain (relative to reference populations). The unit of BII is \% as it is a proportion of species. The NHM's Index is the best for this project as the database used is that of the PREDICTS project, which more geographically representative than other datasets \citep{purvis2018modelling}. This allows for direct comparison of these changes, with the changes in anthropogenic pressures. Historical BII data spanned 18 years between 1970 - 2014. 
	
	% Climate data = 230 countries
	I obtained climate change time series data in the form of annual average temperature for each country. The temperature dataset chosen is from the World Bank's Climate Change Knowledge Portal. I chose this dataset because it contains comprehensive historical data, providing an annual average temperature for every year from 1900 until 2020. 
	
	% Built Land data = 249 countries (years is big problem!) \newline
	I chose data from the Global Human Settlement Layer data package \citep{JRC117104} to represent land use change. The data contains information on built-up area change over time, which is the variable I chose to represent land use change. The dataset provides the total area of built up land (in square kilometres) for various years. Collective land use change is difficult to quantify from land use statistics. Data is available for amount of each land use type over time, and calculating annual land use change from the proportion of each land cover type is not necessarily accurate, as land use change can be multi-directional. Although satellite data is available to categorise land cover type over time. Current studies assessing the impact of land use change on biodiversity are often meta-analyses or use a natural regional situation as the reference land type \citep{de2013land} as opposed to observing direct impacts of land use change. A limitation of this dataset is that only 4 years of data are included. 
	
	% GHG data = 31 years, 63 countries (countries is problem!) \newline
	I chose greenhouse gases (GHG) to represent pollution, as the focus is on terrestrial biodiversity. I chose the dataset to represent GHG emissions for each country over time as the `National Inventory Submissions' section of the United Nations - Climate Change website \citep{UN2022}. GHG emissions in this dataset have a unit of `thousand tonnes of CO2 equivalent'. GHG emissions are presented both including and excluding `Land Use, Land-Use Change and Forestry (LULUCF)' related emissions data. When assessing pollution and biodiversity links in isolation, I included LULUCF. However, when modelling all biodiversity pressures together, I tested LULUCF for collinearity with the land use change variable, and consequently included/excluded. I used the OECD.stat website to download the land use change and pollution data. 
	
	I obtained invasive species data from the Alien Species First Record Database \citep{seebens2017no, seebens2018global}. This database provides information about how many new invasive species were recorded in each country, every year. The reason that I used this database is that it provides time series data. The time span of first records of species spans from 7000BC to 2020.  
	
	I excluded overexploitation from this analysis for multiple reasons surrounding difficulty obtaining data. Firstly, overexploitation is the vaguest of the main pressures, and is usually used in the context of fishing (marine biodiversity being beyond the scope of this project). One of the most relevant aspects of overexploitation to is deforestation, which is (maybe remove this part depending on methods) already represented in the variable used for land use change. Though there are other aspects of overexploitation that would be relevant (e.g. illegal wildlife trading), there are no databases/studies available representing overexploitation by country.

	\subsection*{Data Wrangling}
	\addcontentsline{toc}{subsection}{Data Wrangling} 
	
	Each data set has data from a different combination of years, and countries. For each pressure being investigated, only data from years and countries that are shared between that particular dataset and the biodiversity dataset is included. 
	
	For each pressure, I wrangled and refined the datasets to obtain two time series (at an annual level) for each country; biodiversity and the magnitude of the particular pressure.
	
	% if i decide to Z standardize data, say "i z standardized data to a mean of 0 and a SD of 1 because ..."
	
	% could do one cumulative section about how sensitivity scores were obtained.
	\subsection*{Obtaining Sensitivity Scores}
	\addcontentsline{toc}{subsection}{Obtaining Sensitivity Scores}
	
	For each country, I fit a linear model with biodiversity (BII) as the response variable, and the biodiversity pressure (e.g. pollution) as the explanatory variable. Though both the biodiversity and pressure datasets were time series, I plotted the biodiversity and pressure data points against each other with year being disregarded. I recorded the gradient coefficient as a `sensitivity score', with the standard error of the gradient also recorded, for use in the next step. This sensitivity score is representative of the sensitivity of that country's biodiversity, to the particular biodiversity pressure (e.g., sensitivity of a country's biodiversity to one unit of pollution). The unit of sensitivity scores depend on the pressure being modelled, and represent the \% change in Biodiversity Intactness Index (BII) for a unit increase in pressure.  
	
	I assigned a 'sensitivity score' representing the sensitivity of each country's biodiversity to climate change using linear models. I fit linear models for each country, with average annual temperature ($^\circ C$) as the explanatory variable, and biodiversity intactness index (\%) as the response variable. The sensitivity score (gradient coefficient) therefore had a unit of BII \% $/ ^\circ C$ representing the \% change in BII for each annual average temperature increase of 1$^\circ C$.
	
	I assigned a 'sensitivity score' representing the sensitivity of each country's biodiversity to pollution using linear models. I fit linear models for each country, with tonnes of CO2 equivalent emissions (000s) as the explanatory variable, and biodiversity intactness index (\%) as the response variable. The sensitivity score (gradient coefficient) therefore had a unit of BII \% / tonnes of CO2 equivalent (000s) representing the \% change in BII for each annual additional thousand tonnes of CO2 equivalent emissions. 
	
	I assigned a 'sensitivity score' representing the sensitivity of each country's biodiversity to Land Use Change using linear models. I fit linear models for each country, with area of built up land (square kilometres) as the explanatory variable, and biodiversity intactness index (\%) as the response variable. The sensitivity score (gradient coefficient) therefore had a unit of BII \% / square kilometre (of built up land) representing the \% change in BII for each built up land increase of 1 square kilometre.
	
	I assigned a 'sensitivity score' representing the sensitivity of each country's biodiversity to invasive species using linear models. I fit linear models for each country, with number of new invasive species as the explanatory variable, and biodiversity intactness index (\%) as the response variable. The sensitivity score (gradient coefficient) therefore had a unit of BII \% / new invasive species representing the \% change in BII for each new invasive species. 

	\phantomsection
	\subsection*{Modelling Continental Differences}
	\addcontentsline{toc}{subsection}{Modelling Continental Differences} 
	
	I compared sensitivity scores between continents using a separate linear model for each pressure. Each model used continent as a categorical explanatory variable, and sensitivity scores for that pressures as the response variable. I used the weights function of lm() to take account for the standard errors of the gradients from the first step, and I inputted the inverse of the standard errors as weights (weights in lm() is inversely proportional to variance). I assigned the reference category as the most numerous category \citep{peng2014discrepancy}. I investigated any extreme lone outliers.
	
	\clearpage

	\section*{Results}
	\addcontentsline{toc}{section}{Results}
	 \phantomsection
	 % julia likes degrees of freedom so maybe include these
	 % Julia is interested in the things to check before running a model. might ask me about that
	 %Julia - repeatability? how to calculate?
	 % add in sentence at beginning of each mode sayign we found or ddidn't find continental differences in sensitivity
	\subsection*{Climate Model}
	\addcontentsline{toc}{subsection}{Climate Model}
	
	158 countries and 18 years matched between the climate and biodiversity datasets. The sensitivity scores of countries to climate change had a mean of -0.016 (SD 0.0037, range: -0.24 - 0.078, n = 158), meaning for each increase in average temperature of $1^\circ C$, the average country's BII decreases by 0.016. \autoref{figure:climatebox} groups  country-level climate sensitivity scores by continent and \autoref{figure:climatemap} shows their geographical distribution. 

	\begin{center}
		\includegraphics[width=1in]{../Images/ClimateSensitivityBoxplot.pdf}
		\captionof{figure}{Biodiversity sensitivity to climate change: country sensitivity scores grouped by continent. \textmd{I investigated the extreme lone outlier of Oceania, which was Fiji (-0.079), and there was no reason to exclude this country from the data as its' standard error was within the range of other countries in Oceania.}}
		\label{figure:climatebox}
		
		\includegraphics[width=7in]{../Images/ClimateSensitivityMap.pdf}
		\captionof{figure}{Biodiversity sensitivity to climate change: distribution of country sensitivity scores.}
		\label{figure:climatemap}


	\end{center}
	\bigskip
	
	The mean sensitivity score for Africa (the reference category) did not significantly different from zero, and sensitivity scores in other continents were not significantly different from Africa's apart from Europe and South America (\autoref{tab:climatetable}). For every $1^\circ C$ increase in annual temperature, the BII of countries within Europe will decrease by 0.008 less than in other continents (Excluding South America), and the BII of countries within South America will decrease by 0.013 more than in other continents.
	% Ask James: Do you say in results whether findings support hypothesis? Is it right that I have presented the gradients as "this much more than other continents", I know that it means this gradient plus the reference category

	\newpage
	
	\begin{table}[H]
		\begin{center}
			\caption{Results from a linear model explaining effects of Continent on Climate Change Sensitivity Scores. \textmd{Statistically significant effects are in bold. N = 158.}}
			\label{tab:climatetable}
			\begin{tabular}{l|r|r|r|r}
				Variable & $\beta$ & \textit{SE} & \textit{t} & \textit{p}\\
				\hline
				Intercept (reference: Africa) & $<$ -0.001 & $<$ 0.001 & 0 & 1\\
				Asia & $<$ 0.001 & $<$ 0.001& 0 & 1\\
				\textbf{Europe} & \textbf{0.008} & \textbf{0.003} & \textbf{2.61} & \textbf{0.01} \\
				North America & $<$ -0.001 & $<$ 0.001 & -0.022 & 0.98\\
				Oceania & -0.003 & 0.007 & -0.39 & 0.70\\
				\textbf{S America} & \textbf{-0.013} & \textbf{0.005} & \textbf{-2.59} & \textbf{0.01}\\
				% Ask James to check sig figs and decimal places
			\end{tabular}
		\end{center}
	\end{table}
	

	\newpage
	
	\subsection*{Pollution Model}
	\addcontentsline{toc}{subsection}{Pollution Model}
	
	There were only 43 countries and 16 years that matched between the datasets (with no data for any African countries). I removed one country with a standard error of 0 due to the inability to calculate variance (weights function uses 1/SE). I investigated two extreme lone outliers in Europe; Spain(-0.000304) and Switzerland(-0.0000704). Both of which had standard errors approximately 6x that of the next largest standard error, and therefore I removed both. The remaining sensitivity scores of countries (n = 40) to climate change had a mean of -0.000000274 (SE 0.000000447, range: -0.00000601  0.0000138), so for each thousand tonne increase of CO2 equivalent emissions, the average country's BII decreases by 0.000000274. \autoref{figure:pollutionbox} groups the country-level pollution sensitivity scores by continent, and \autoref{figure:pollutionmap} shows their geographical distribution. \newline

	\begin{center}
		
		\includegraphics[width=7in]{../Images/PollutionSensitivityBoxplot.pdf}
		\captionof{figure}{Biodiversity sensitivity to greenhouse gas emissions: country sensitivity scores grouped by continent.}
		\label{figure:pollutionbox}
		
		\includegraphics[width=7in]{../Images/PollutionSensitivityMap.pdf}
		\captionof{figure}{Biodiversity sensitivity to pollution.}
		\label{figure:pollutionmap}
		

	\end{center}
	The average sensitivity score for countries in Asia (the reference category) did not significantly different from zero, and sensitivity scores in other continents were not significantly different from Asia's (\autoref{tab:pollution}).
	%Ask James: do I need to say more than this?
	
		\begin{table}[H]
		\begin{center}
			\caption{Results from a linear model explaining effects of Continent on Pollution Sensitivity Scores. \textmd{ N=42.}}
			\label{tab:pollution}
			\begin{tabular}{l|r|r|r|r} % <-- Alignments: 1st column left, 2nd middle and 3rd right, with vertical lines in between
				Variable & $\beta$ & \textit{SE} & \textit{t} & \textit{p}\\
				\hline
				Intercept (reference: Europe) & $<$ -0.001 & $<$0.001 & -0.042 & 0.967\\
				Asia & $<$ 0.001 & $<$0.001 & 1.449 & 0.156 \\
				North America & $<$ 0.001 & $<$0.001 & 0.099 & 0.922\\
				Oceania & $<$ 0.001 & $<$0.001 & 0.397 & 0.694\\
				S America & $<$ -0.001 & $<$0.001 & -0.335 & 0.739\\
				% Ask James to check sig figs and decimal places
			\end{tabular}
		\end{center}
	\end{table}

\newpage
	\subsection*{Land Use Model}
	\addcontentsline{toc}{subsection}{Land Use Model}
	
	175 countries and 3 years matched between datasets. Eight countries with a standard error of 0 were removed due to the inability to calculate variance (weights function uses 1/SE). Sensitivity scores of Europe, Oceania and South America each had extreme lone outliers which were removed. Europe's outlier was Ireland (2.756) which had a standard error of 2.108, which was 100x that of the next highest standard error in Europe. Although this high standard errors are already accounted for in the model, I removed Ireland due to the scale of difference to the rest of the data within Europe. South America's outlier was Honduras (-3.223) with a standard error of 2.466. Due to the standard error of Honduras being 1000x that of the next highest standard error of South American countries, Honduras was also excluded. Oceania's outlier was Samoa with a standard error of 2.344, which was 100x larger than that of the next highest standard error in Oceania, and therefore was removed.
	
	 The sensitivity scores of remaining countries (n = 164) to built up land had a mean of -0.0006 (SE 0.0009, range: -0.12 to 0.07), so for each square kilometre increase in built up land area, the average country's BII decreases by 0.0006. \autoref{fig:landusebox} groups  country-level built up land sensitivity scores by continent, and \autoref{fig:landusemap} shows their geographical distribution.
	
	\begin{center}
		
		\includegraphics[width=7in]{../Images/LandUseSensitivityBoxplot.pdf}
		\captionof{figure}{Biodiversity sensitivity to built up land use: country sensitivity scores grouped by continent.}
		\label{fig:landusebox}
		
		\includegraphics[width=7in]{../Images/LandUseSensitivityMap.pdf}
		\captionof{figure}{Biodiversity sensitivity to build up land.}
		\label{fig:landusemap}
		

    \end{center}

    The average sensitivity score for countries in Africa (the reference category) did not significantly different from zero, and sensitivity scores in other continents were not significantly different from Africa's (\autoref{tab:landuse}).
    
    \begin{table}[H]
    	\begin{center}
    		\caption{Results from a linear model explaining effects of continent on land use sensitivity scores. \textmd{N=167.}}
    		\label{tab:landuse}
    		\begin{tabular}{l|r|r|r|r} % <-- Alignments: 1st column left, 2nd middle and 3rd right, with vertical lines in between
    			Variable & $\beta$ & \textit{SE} & \textit{t} & \textit{p}\\
    			\hline
    			Intercept (reference: Africa) & $<$ -0.001 & $<$ 0.001 & -1.039 & 0.301\\
    			Asia & $<$ 0.001 & $<$ 0.001 & 0.374 & 0.709\\
    			Europe & $<$ 0.001 & $<$ 0.001 & 0.902 & 0.369 \\
    			North America & $<$ 0.001 & $<$ 0.001 & 0.638 & 0.524\\
    			Oceania & $<$ 0.001 & $<$ 0.001 & 0.012 & 0.990\\
    			S America & $<$ -0.001 & $<$ 0.001 & -0.341 & 0.734\\
    			% Ask James to check sig figs and decimal places
    		\end{tabular}
    	\end{center}
    \end{table}
    
    \newpage
	\subsection*{Invasive Species Model}
	\addcontentsline{toc}{subsection}{Invasive Species Model}

	
	91 countries matched between datasets, with a different amount of years matching per country. 3 countries with a standard error of 0 were removed. I investigated the two extreme lone outliers in Europe  (Bosnia and Herzegovina) and Africa (Malawi). I found no reason to exclude either, as although both had one of the highest standard error of their respective continents, the standard errors were not drastically different from others within the continent. Additionally, standard error is already accounted for with the weights function, meaning these outliers already carry less weight in the model than other countries in the same continents.
	
	The sensitivity scores of countries (n = 91) to invasive species had a mean of 0.0017 (SD 0.0012, range: -0.023  0.055), so for each additional new invasive species, the average country's BII increases by 0.0017. \autoref{invasivebox} groups country-level invasive species sensitivity scores by continent, and \autoref{invasivemap} shows their distribution. \newline

	\begin{center}
		
		\includegraphics[width=7in]{../Images/InvasiveSpeciesSensitivityBoxplot.pdf}
		\captionof{figure}{Biodiversity sensitivity to invasive species: country sensitivity scores grouped by continent. \textmd{Extreme lone outliers include Bosnia and Herzegovina (0.041) for Europe and Malawi (0.055), both of which were kept in for modelling.}}
		\label{invasivebox}
		
		\includegraphics[width=7in]{../Images/InvasiveSpeciesSensitivityMap.pdf}
		\captionof{figure}{Biodiversity sensitivity to Invasive species.}
		\label{invasivemap}
		

	\end{center}

    The average sensitivity score for countries in Europe (the reference category) did not significantly different from zero, and sensitivity scores in other continents were not significantly different from Africa's (\autoref{tab:invasive}). 
    
    
	\begin{table}[H]
		\begin{center}
			\caption{Results from a linear model explaining effects of continent on invasive species sensitivity scores.\textmd{ N=91.}}
			\label{tab:invasive}
			\begin{tabular}{l|r|r|r|r} % <-- Alignments: 1st column left, 2nd middle and 3rd right, with vertical lines in between
				Variable & $\beta$ & \textit{SE} & \textit{t} & \textit{p}\\
				\hline
				Intercept (reference: Europe) & $<$ -0.001 & $<$ 0.001 & -0.678 & 0.5\\
				Africa & $<$ 0.001 & $<$ 0.001 & 0.678 & 0.5\\
				Asia & $<$ 0.001 & $<$ 0.001 & -0.67 & 0.5 \\
				North America & $<$ 0.001 & 0.001 & 0.966 & 0.337\\
				Oceania & $<$ -0.001 & $<$ 0.001 & -0.216 & 0.83\\
				
				% Ask James to check sig figs and decimal places
			\end{tabular}
		\end{center}
	\end{table}

    \clearpage
    
     \section*{Discussion}
     \addcontentsline{toc}{section}{Discussion}
     % this first sentence is too confusing
      This study found that there was continental variation in biodiversity sensitivity in response to climate change, but not in response to the other three pressures (land use change, pollution and invasive species). 
     
     \phantomsection
     \subsection*{Climate Model}
     \addcontentsline{toc}{subsection}{Climate Model}
     
     The climate change results support the biome concept, that  differing biome sensitivities being the cause of continent-wide differences. Although the effect sizes might seem small, the BII expresses biodiversity as a proportion of species that remain in comparison to the original natural population \citep{phillips2021}, and therefore even a seemingly small change in BII may be biologically meaningful as it may represent many species.

     % can say differences could be because of different proportions of tropical and temperate biomes
     Geographical differences in sensitivity to climate change could result from variations in historical disturbance of each area. Areas of higher historical disturbance have been shown to be less sensitive to pressures \citep{newbold2020tropical, willmer2022global}, % so is something that could be included in further modelling. put this in further work
     
     geographical variations in biodiversity sensitivity have been shown to correlate with climactic seasonality.
     Climate seasonality is something that has correlated with regional variations in biodiversity sensitivity \citep{newbold2016global, newbold2020tropical}, and could be the cause of variation in my findings.
     
     *Compare results with the one study that looked at biome differences between continents*.
    \phantomsection
	\subsection*{Land Use Change Model}
	\addcontentsline{toc}{subsection}{Land Use Change Model}
    Continents showed no variation in their sensitivity to land use change (represented by change in built land). These results are conflicting with those of another study which found that response to land use varied significantly between continents \citep{gibson2011primary, phillips2016effects}. \citet{gibson2011primary} and speculated that differences could be due to intrinsic differences in the sensitivity of continents' biodiversity. 
     \phantomsection
     \subsection*{Pollution Model}
     \addcontentsline{toc}{subsection}{Pollution Model}
     
     
    \phantomsection
    \subsection*{Invasive Species Model}
    \addcontentsline{toc}{subsection}{Invasive Species Model}
    
	\phantomsection
	\subsection*{Non-Significant Models}
	\addcontentsline{toc}{subsection}{Non-Significant Models}
     
     talk in discussion about no relationship ebtween some pressures and biodiversity. Is there delay in repsonse? Do these indicators not capture true effect pressures are having
     
     could be because biomes themselves actually don't vary much in sensitivity, as supported by \citet{newbold2015global}.
     
     My findings were not in line with theories put forward about geographical differences in biodiversity change being affected by geographical differences in biodiversity sensitivity \citep{blowes2019geography}.
     
	\phantomsection
	\subsection*{Pressure Interactions}
	\addcontentsline{toc}{subsection}{Pressure Interactions}
	
	Though there was not sufficient data to allow all pressures to be modelled together, as explained earlier, modelling pressures separately means important interactions \citep{hansen2001global} were not accounted for in this study. The assumptions made by modelling pressures separately is that the impacts of pressures on biodiversity are additive (no interactions),  however there is also the possibility that the pressures interact synergistically (stronger when together) or antagonistically (weaker when together) \citep{bowler2020mapping}. 
	
	\phantomsection
	\subsection*{Data Limitations} % combine sections into "limitations and future work" but still have them as separate paragraphs
	\addcontentsline{toc}{subsection}{Data Limitations}	
	
	How representative are the variables I used to represent pressures, of the actual pressures themselves?
	How representative is BII, of the actual biodiversity? Same concept was discussed in \citet{goethem2021biodiversity} and \citet{secretariat2014global} (says it is difficult to represent biodiversity with one metric). Further research should look into this and perhaps use alternative measures of biodiversity to test whether trends differ. 
	The BII could also show geographic bias in the data used to calculate each country's BII. Tropical realms are often under-represented \citep{collen2008tropical}. It is for this reason that the PREDICTS database, which the BII values are based on, was designed for testing local-scale biodiversity hypotheses, and so further research with alternative datasets should be done to test the findings of this study.
	
	After datasets had been merged, the years of data which matched and were omnipresent were often not successive years. The years included in the BII dataset limited the amount of years that had a potential to match across pressures as data was in 10 year intervals until 2000, as opposed to yearly intervals after 2000. This sparsity of BII data prior to 2000 could have limited the statistical power of results. The National History Museum will soon update the dataset of BII values that I used to represent biodiversity, which will update the most recent year from 2014 (as used in this study) to 2021 \citep{nhm2021}. Further analysis after release of this data could result in more common data between datasets and could show new insights.
	
	It would be more useful if we had a better variable to represent land use change. Habitat loss is the most pervasive driver of terrestrial biodiversity loss (there is a reference for this somewhere in this notebook, world something alliance idk), and there is no database quatifying land use change. There might be values for land use, and there might be satellite data that visualises change over time, but access to data with numerical values representing land use change over time  is limited (there is actually one in OECD website so if I don't have time to change it to this, mention it). 
	
	Invasive species data includes whether there was a new invasive species that year but doesn't include the distribution of the invasive species (could be widespread or barely there). If I haven't done so by now, I think I should use an accumulative value for invasive species instead of new species that year?
	
	
	\phantomsection
	\subsection*{Method Limitations}
	\addcontentsline{toc}{subsection}{Method Limitations}
     
     To my knowledge the only other study which obtained country-level sensitivity scores  \citep{louette2010bioscore} did so by summing how many species within that country were classed as 'negatively impacted' by biodiversity pressures, according to prior literature and experts. For aforementioned reasons, I decided to take an alternative approach in the interest of obtaining results at the global level. I analysed relationships between country averages of biodiversity and its' main pressures, to infer the sensitivity of the country by interpreting the gradient as a sensitivity score. Obtaining sensitivity scores implicitly in this way could have limited their accuracy, and further research comparing the two methods would provide more insight.
     
	Important to consider that human pressures are not the only drivers of biodiversity change, as has been assumed in this study. Other factors to consider in future study could be relative degree of conservation effort which could be slowing biodiversity loss despite pressures. Other limitations regarding pressures is that although this method acknowledges that each country experiences varying magnitudes of pressures, it assumes the most relevant pressures for each country are the same globally. Individual countries each use a different group of pressures to model their biodiversity trends \citep{henly2021biodiversity} which could suggest that using the same group of pressures globally is missing out vital information.
	
     
     \phantomsection
     \subsection*{Further Study}
     \addcontentsline{toc}{subsection}{Further Study}
     
     A possible reason for not finding more support for the hypothesis is that each biome's sensitivity could differ between continents, and not be constant as was assumed by the hypothesis. Tropical biomes have been shown to be more sensitive in Asia than in Africa and the Americas \citep{gibson2011primary}, however to my knowledge there is only one study about geographical variations in biome sensitivity, and it was limited to one biome, being the reason for exclusion from the hypothesis reasoning. Further research into biome sensitivity, and in particular how it differs according to continent, could provide more insight into how continent biodiversity sensitivity differs. 
     

     
     Further research should attempt to quantify and include effects of overexploitation, which was not included in this study. 
     
    
         
    \phantomsection
    \subsection*{Applications} % combine with conclusions
    \addcontentsline{toc}{subsection}{Uses of Findings}
    %Para: Sustainable business and investments
    Whilst the conservation movement does a lot to help biodiversity \citep{sandbrook2019global}, another response to the biodiversity crisis \citep{ogar2020science} is the beginning of a global movement towards sustainable business and biodiversity-conscious investment \citep{pri2020, worldeconomicforum2020, wwf2020}. Creating a tool which assesses the overall biodiversity impact of a company can help guide investors, and is something that various parties are currently developing \citep{worldbenchmarkingalliance_2022, iccs_2020}. `Benchmark for Nature' is a project which aims to use data science and to develop a framework for assessing investment impacts on biodiversity by gathering information from online articles about how much of an impact a company/sector is having on biodiversity pressures\citep{iccs_2020}, and where in the world these pressures are taking place. 
    \clearpage
    
    \phantomsection
    \subsection*{Conclusion}
    \addcontentsline{toc}{subsection}{Conclusion}
    hold back one perspective to mention in conclusion
    Uncertainty of these results is considerable.
    Can say  how I think the solution to our biodiversity problem isn’t going to be human willpower but instead some advances in technology will save the day    
    
    An understanding of this geographical variation (or lack thereof) will improve the accuracy of predictions about biodiversity response to pressures % say why.
    
    \section*{Data and Code Availability}
    \addcontentsline{toc}{section}{Data and Code Availability}
    Data is available on \href{https://www.dropbox.com/sh/pnoxzydwgmj4eaf/AABYpBAkeQJsY3yqGH8G8J86a?dl=0}{Dropbox.} 
    Code is available on GitHub in the 'Project' folder of the \href{https://github.com/kayleigh-greenwood/CMEECoursework.git}{Coursework Repository.}
    
    \clearpage
    \section*{Acknowledgements}
    \addcontentsline{toc}{section}{Acknowledgements}
    I would like express my deepest appreciation to my supervisor Dr James Rosindell for his support and feedback throughout this project. I would also like to thank my external supervisor Dr Joss Wright for his advice and expertise. I am also grateful to Dr Will Pearse for generously providing extensive advice regarding the statistics of the project. I also gratefully acknowledge the help of Isabella Brizzolara in facilitating my access to certain required material.
    
    Lastly, I extend my special thanks to Greig Mann for his unwavering support and feedback throughout this project.
    
    \newpage
    \bibliographystyle{apalike}
    \bibliography{writeup}

\end{document}