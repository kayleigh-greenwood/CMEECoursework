\documentclass[11pt, a4paper, titlepage]{article}
\usepackage[onehalfspacing]{setspace} 
\usepackage{fullpage}
\usepackage{helvet}
\renewcommand{\familydefault}{\sfdefault}
\usepackage{graphicx}
\usepackage{float} % needed for [H]
\usepackage{titlesec}
\usepackage{lineno}
\usepackage[margin=2cm]{geometry}
\linenumbers
\onehalfspacing
\usepackage{pgfgantt}


\begin{document}
    \begin{titlepage}
    \begin{center}
            {\large IMPERIAL COLLEGE LONDON}
    \end{center}
    
    \vspace*{\fill}
    
    \begin{center}
        {\Huge Modelling the impact of geography on a sector's biodiversity impact}
        \\[2in]
        Author: Kayleigh Greenwood, MSc CMEE (kg21@ic.ac.uk)
        \bigskip
        \newline
       Internal Supervisor: Dr James Rosindell, Imperial College London (j.rosindell@imperial.ac.uk)
       \bigskip
       \newline
        External Supervisor: Dr Joss Wright, University of Oxford (joss.wright@oii.ox.ac.uk)
        \bigskip
        \newline

        04/04/2022
        \\[2in]
        
        {\bfseries Keywords: Biodiversity, Mathematical Modelling, Geography, Sustainable Investment, News Article Data, biodiversity drivers, anthropocene }

        

    \end{center}
    
    \vspace{\fill}
    
    \end{titlepage}

    \section*{Introduction}
    
   	Benchmark for Nature is a project at the University of Oxford's Internet Institute, emerging from the global movement towards sustainable business. The project aims to use data science to develop a framework for assessing investment impacts on biodiversity \cite{iccs_2020}, similar to the ESG framework currently in place. The ESG information currently available only assesses an investments' impact on the environment, but not on living nature and biodiversity. The purpose of the project is to better inform investors so that more biodiversity-conscious decisions can be made, in an effort to aid the biodiversity loss crisis \cite{gasu2021review}, and also indirectly, the climate \cite{shin2022actions}.
   	\newline
   	\newline
   	Many indicators have been determined within the project as definitions of how biodiversity impact will be assessed. One question that has been raised is whether these indicators have equal impacts on biodiversity regardless of geography, as is a current assumption. Studies show impacts of socioeconomic status and cultural impacts on biodiversity \cite{kinzig2005effects}. This gives reason to believe that contributors towards biodiversity loss could have varying impacts based on their location. For example, this research question would aim to provide answers for whether investing in a specific sector in one country could have a different impact on biodiversity than investing in the same sector in another country. I will aim to address this question and determine any association between geography and the level of biodiversity impact of a company.


    \section*{Methods}

	Benchmark for nature is working on open-source data from news articles that have been scraped over the past few years. Therefore my project involves no data collection, and I will be working with data that is already available.
	\newline
	\newline
	My project will involve pulling out geographical elements from this data and devising a method that can assess how impacts change based on country. It is important to note that the research question is not about the impact of country on biodiversity, but rather if country can affect the scale of a sector's impact on biodiversity. I will use a generalised linear model, with country as an explanatory variable. The response variable will be specified upon further inspection of the data but will roughly be associated with how often a sector in that country is associated with indicators of biodiversity loss. 
	

	\clearpage
	\section*{Outcomes}
	
	The main stakeholder in my project is the Benchmark for Nature team, who are interested in the answer to the question I am raising. There has been no work done as of yet within the project to determine geographical differences, therefore the outcome of my project will aid the researchers on the project in their research into geographical effects of investments on biodiversity.
	
	Figure 1 shows a timeline of the projects' progress
	\\[0.5in]
	\begin{ganttchart}{1}{21}
		\gantttitle{Figure 1. Project Timeline}{21} \\
		\gantttitlelist{1,...,21}{1} \\
		\ganttbar{Decide hypothesis}{1}{1} \\
		\ganttbar{Read literature}{2}{4} \\
		\ganttbar{Introduction}{4}{5} \\
		\ganttbar{Modelling}{6}{15} \\
		\ganttbar{Methods}{6}{15} \\
		\ganttbar{Results}{16}{16} \\
		\ganttbar{Discussion}{17}{19} \\
		\ganttbar{Finalising Writeup}{19}{21}
	\end{ganttchart}
	

	\section*{Feasibility}
	

	This project is very feasible, given that the data has already been collected and the question has many gradations of success.
	\newline
	There is no data collection involved, therefore no funding is needed for field or lab work. There may be small costs involved for software access, given that this is a desk based project. 



    \clearpage
    \bibliographystyle{apalike}

    \bibliography{Proposal}
\end{document}