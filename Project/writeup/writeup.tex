\documentclass[11pt, a4paper, titlepage]{article}
\usepackage[onehalfspacing]{setspace} 
\usepackage{fullpage}
\usepackage{helvet}
\renewcommand{\familydefault}{\sfdefault}
\usepackage{graphicx}
\usepackage{float} % needed for [H]
\usepackage{titlesec}
\usepackage{lineno}
\usepackage[margin=2cm]{geometry}
\linenumbers
\onehalfspacing
\usepackage{pgfgantt}


\begin{document}
    \begin{titlepage}
    \begin{center}
            {\large IMPERIAL COLLEGE LONDON}
    \end{center}
    
    \vspace*{\fill}
    
    \begin{center}
        {\Huge 
    	 Geographical variations in the sensitivity of terrestrial biodiversity to anthropogenic pressures}
        \\[2in]
        Author: Kayleigh Greenwood, MSc CMEE (kg21@ic.ac.uk)
        \bigskip
        \newline
       Internal Supervisor: Dr James Rosindell, Imperial College London (j.rosindell@imperial.ac.uk)
       \bigskip
       \newline
        External Supervisor: Dr Joss Wright, University of Oxford (joss.wright@oii.ox.ac.uk)
        \bigskip
        \newline

        25/08/2022
        \\[2in]
        
        {\bfseries A thesis submitted in partial fulfilment of the requirements for the degree of Master of Science at Imperial College London \newline \newline Submitted for the MSc in Computational Methods in Ecology and Evolution }

        

    
	\end{center}
    \vspace{\fill}
    
    \end{titlepage}
	\section*{Declaration}
	\begin{center}
	Was the data provided to you or did you collect or assemble it? \newline
	Were you responsible for data processing or cleaning, if required? \newline
	Were any mathematical models developed by you or by your supervisor? \newline
	What role, if any, did your supervisor play in developing the analyses presented?
	\end{center}

	\newpage
    \section*{Introduction}
    
   	 Biodiversity is important because it supports life on earth via the ecosystem services provided. When ecosystems have their biodiversity intact, they can provide services such as clean air and pollination, which makes the earth habitable for humans. Biodiversity loss leads to unstable environments, as ecosystems with low biodiversity are less resistant to change. Biodiversity loss diminishes ecosystem productivity \cite{duffy2017biodiversity} and threatens human well-being \cite{diaz2006biodiversity}. \newline
   	 
   	 Biodiversity is impacted by both natural and anthropogenic pressures \cite{nobel2020anthropogenic}, however any mention of 'biodiversity pressures' in this study refers only to the latter. Understanding the impacts of anthropogenic pressures on biodiversity is important for creating accurate policies and conservation strategies. Accurate information about biodiversity pressures can produce more effective conservation strategies, and better informed decisions can be made, including biodiversity-conscious investments. One of many responses to the biodiversity crisis \cite{ogar2020science} is the beginning of a global movement towards sustainable business and biodiversity-conscious investment \cite{pri2020}\cite{worldeconomicforum2020}\cite{wwf2020}. \newline
   	 
   	 Assessing the impact that investments have on biodiversity involves calculating the magnitude of association they have with each biodiversity pressure. Information is often available on the geography of a company's activities, such as where they base their factories or where they source their materials from. In the interest of making estimates about each company's biodiversity impact more accurate, the location of each company's biodiversity-related activities could be considered. If the location and magnitude of a biodiversity pressure is provided (by the company), then information about current local biodiversity and sensitivity of the biome to the pressure, accurate predictions could be made about how biodiversity-friendly such an investment would be.  \newline
   	 
   	 Given that anthropogenic impact on the environment is worldwide \cite{plumptre2021might}, the question should be raised of whether the geographic location of biodiversity pressures affects their impact on global biodiversity. In other words, are some parts of the world more sensitive to biodiversity pressures than others? For example, does the location that a biodiversity pressure takes place change its impact on global biodiversity (regardless of magnitude)? If such geographic differences exist, they should be taken into account when attributing biodiversity-related merit to investments. Better understanding of biodiversity pressures will aid a better understanding of the implications of investments on natural ecosystems . \newline
   	
   	\newpage
   	\section*{Literature Review}
   	Literature is abundant on how biodiversity varies by country {reference}, and why this may be, including how various direct and indirect pressures correlate with these differences \cite{sunday2015species}, \cite{ament2019compatibility} {melusine's paper}. 
	Various studies have mapped the magnitude of biodiversity pressures across regions/biomes \newline \cite{millennium2005ecosystems} \cite{sala2000global}, and their spatial couplings \cite{bowler2020mapping}, however to our knowledge, no prior research has studied geographic differences in sensitivity to such pressures. Bowler et al.(2020) concluded that despite any patterns observed in magnitude of pressures, there will always be variation in biodiversity response to such pressures due to species' varying sensitivities. Research about species-specific sensitivities in each ecosystem is useful for local conservation policy however it would be more useful for large scale projects/policies to have information about the sensitivity of regions/biomes on the whole. The current assumption in literature is that whilst magnitude of exposure varies, sensitivity to biodiversity pressures is constant across biomes \cite{sala2000global}, however there is no research to support this assumption. Hence, studying variation in biome sensitivity would be useful in comparing the impact of pressures in these areas on global biodiversity.  \newline
	
	There is adequate research to prove that inter-species responses to biodiversity pressures vary {find reference}.Given that species vary in their sensitivity, and therefore response, to biodiversity pressures, and that each region of the world comprises different combinations of species groups {find reference}, there is reason to believe that sensitivity to biodiversity pressure could vary depending on region. \newline
	
	One of the papers which studied sensitivity of species to environmental pressures \cite{louette2010bioscore}, developed a set of sensitivity scores for European species, determining which species will benefit from, be indifferent to, or be negatively affected by environmental change. This 'Bioscore' study used such sensitivity scores to create a tool for predicting the effect of a policy change on Europe's biodiversity. The proportion of  affected species in each region was used to map the effects of a change in each biodiversity pressure. The sensitivity scores for each species were obtained from published literature about individual species' responses to change in different environmental variables. The BioScore tool suggests that even if the magnitude of a biodiversity pressure is constant across Europe, biodiversity's response can still vary according to country, due to varying sensitivity of the species within such country. This study is a predictive tool based on published studies about individual species, and a wider-breadth study is necessary to observe worldwide variances in countries sensitivities to biodiversity pressures. The BioScore tool's predictions support the concept that country-wide differences in sensitivity could exist. \newline
	
	A wider spectrum study examined sensitivity to environmental change at a broader level, and found variation between taxa \cite{sunday2015species}. This between-taxa variation further supports the concept that sensitivity to biodiversity pressures could vary between countries, and the authors emphasise that their findings suggest that sensitivity to environmental change should not be assumed to be constant across taxa, as is currently common. This supports the idea that researching differences between countries' sensitivity, could contribute to more accurate predictions of how biodiversity pressures impact biodiversity. \newline
	
	
	
	
	
	
	
	
	 
   	 
   	 
   	 ******* \newline
   	 Studies show impacts of socioeconomic status and cultural impacts on biodiversity \cite{kinzig2005effects}. This gives reason to believe that pressures impacting biodiversity loss could have varying impacts based on their location. This research aims to investigate whether the location of a pressure affects its' level of impact on biodiversity. 


    \section*{Methods}


	\subsection*{Data}
	
	The focus of this study is on anthropogenic biodiversity pressures only. Anthropogenic pressures on biodiversity are typically grouped, in the current literature, into 5 main pressures; climate change, land use change, pollution, invasive species and overexploitation. In order to assess whether sensitivity to each pressure varies by country, data was needed in the form of time series (how each of these pressures had been changing in each country over time, as well as how each country's biodiversity had been changing over time). The time series of biodiversity in a country was compared to the time series of a pressure on biodiversity in that country, in order to extract a 'sensitivity score' for each country to assess any effect of geography. \newline
	
	The variable chosen to represent biodiversity was biodiversity intactness. The National History Museum's (NHM) Biodiversity Intactness Index (BII) was chosen as it presents biodiversity in the context of how many original species remain (relative to reference populations). The NHM's Index is the best for this project as the database used is that of the PREDICTS project, which more geographically representative than other datasets \cite{purvis2018modelling}. This allows for direct comparison of these changes, with the changes in anthropogenic pressures. Historical BII data spanned 1970 - 2014. \newline
	
	Time series data for climate change was obtained in the form of annual average temperature for each country. The temperature dataset chosen was from the World Bank's Climate Change Knowledge Portal. This dataset was chosen because it contains comprehensive historical data, providing an annual average temperature for every year from 1900 until 2020. \newline
	
	To represent land use change, the dataset used was The Global Human Settlement Layer data package \cite{JRC117104}. The data contains information on built-up area change over time, which is the variable chosen to represent land use change. Collective land use change is difficult to quantify from land use statistics. Although satellite data is available to categorise land cover type over time, calculating annual land use change from the proportion of each land cover type is not necessarily accurate, as land use change can be multi-directional. Current studies assessing the impact of land use change on biodiversity are often meta-analyses or use a natural regional situation as the reference land type \cite{de2013land} as opposed to observing direct impacts of land use change. Statistics \newline
	
	With the focus being on terrestrial biodiversity, greenhouse gases (GHG) were used as the representative variable for 'pollution' as a biodiversity pressure. The dataset used to access GHG emissions for each country over time was the 'National Inventory Submissions' section of the United Nations - Climate Change website \cite{united nations}. GHG emissions are presented both including and excluding 'Land Use, Land-Use Change and Forestry (LULUCF)' related emissions data. When assessing pollution and biodiversity links in isolation, LULUCF was included. However, when modelling all biodiversity pressures together, LULUCF was tested for collinearity with the land use change variable, and consequently included/excluded. \newline
	
	The OECD.stat website was used to download the land use change and pollution data. \newline

	

	
	\subsection*{Individual Pressures}
	
	For each biodiversity pressure, a linear model was created for each country using time series of biodiversity data and the corresponding pressure's time series. For each country in which the pressure was found to have a significant effect on biodiversity (p<0.05), the sensitivity score was noted, representing such country's 'sensitivity' to this particular biodiversity pressure. \newline
	
	First, each pressure's geographic relationship with biodiversity was assessed in isolation. It is important to look at individual biodiversity pressures, as opposed to an aggregated pressure on biodiversity, because the pressures have spatial differences \cite{steffen2015planetary}, meaning the geographical magnitude of each pressure varies. Therefore in order to understand how countrys differ in their responses to biodiversity pressures, it must be taken into account the magnitude of each pressure that each country experiences. \newline
	
	Each data set has data from a different combination of years, and countries. For each pressure being investigated, only data from years and countries that are shared between that particular dataset and the biodiversity dataset is included. \newline
	
	For each pressure, the datasets were wrangled and refined to obtain two time series (at an annual level) for each country; biodiversity and the magnitude of the particular pressure. 
	
	# old method For each country, a linear model was fit with biodiversity as the response variable, and the biodiversity pressure (e.g. pollution) as the explanatory variable. For those countries where the gradient was found to be statistically significant (p<0.05), the gradient was recorded as a 'sensitivity score'. This sensitivity score is representative of the sensitivity of that country's biodiversity, to the particular biodiversity pressure (e.g., sensitivity of Spain's biodiversity to one unit of pollution) \newline
	
	# old method Sensitivity scores were then used to visualise the differences between countries. (I could insert a map with a colour scale representing the sensitivity score values from different countries) \newline
	
	# new method. The data for all countries was pooled into one dataset, and a column added for continent. Because assessing differences in each country would remove too many degrees of freedom, differences between the sensitivities of continents were assessed. A multiple linear model was created for each pressure. Continent was coded as a factor, in order for R to treat it as a dummy variable. The alphabetically first continent acting as the reference variable (usually Africa), in order to avoid multicollinearity. So that the slopes of each continent could be compared, interactions were also added between continent and the climate
	
	
	
	\subsection*{Multi-Pressure Model}
	
	
	
		
	 
	
	
	

	\clearpage

	 \section*{Results}
	 
	 \subsection*{Individual Pressures}
	 
	 
	 \subsection*{Multi-Pressure Model}
	 
	 

    \clearpage
    
     \section*{Discussion}
     
     \clearpage
     
     \section*{Conclusion }
     optional section
     \clearpage
    
    \section*{Data and Code Availability}
    Data  and  CodeAvailabilitystatement:  At  the  end  of  your  Main  text,  before  the  References section, you must provide a statement titled “Data and Code Availability”, where you name a data (e.g., Dropbox, FigShare, Zenodo, etc) and a code (e.g., Dropbox, GitHub, etc.) archive 
    20from where the data and code can be obtained that will allow replication of your results. The code may be in the form of a single script file.
    
    \clearpage
    \bibliographystyle{apalike}

    \bibliography{writeup}
\end{document}